\documentclass[conference]{IEEEtran}
\IEEEoverridecommandlockouts
% The preceding line is only needed to identify funding in the first footnote. If that is unneeded, please comment it out.
\usepackage{cite}
\usepackage{amsmath,amssymb,amsfonts}
\usepackage{algorithmic}
\usepackage{graphicx}
\usepackage{textcomp}
\usepackage{xcolor}
\def\BibTeX{{\rm B\kern-.05em{\sc i\kern-.025em b}\kern-.08em
    T\kern-.1667em\lower.7ex\hbox{E}\kern-.125emX}}
\begin{document}

\title{RFID Sicherheit}


\author{\IEEEauthorblockN{Julian Hoever}
\IEEEauthorblockA{\textit{Verteilte Systeme} \\
\textit{Universität Duisburg-Essen}\\
Duisburg, Deutschland\\
julian.hoever@stud.uni-due.de}
}

\maketitle

\begin{abstract}
Die folgende Arbeit behandelt das Thema der RFID Sicherheit im Bezug auf Sicherheitslücken, Schutzmaßnahmen und Privatsphäre. Es werden einige mögliche Schwachstellen der RFID/NFC Technik aufgezeigt und Angriffstechniken vorgestellt, welche die zuvor aufgeführten Schwachstellen ausnutzen. Dabei wird darauf eingegangen, in welchen realen Szenarien diese Angriffe eine Bedrohung darstellen, wie zum Beispiel beim kontaktlosen Bezahlen oder der Diebstahlsicherung von Waren. Anschließend werden einige Schutzmaßnahmen skizziert, welche die zuvor genannten Angriffstechniken abmildern oder verhindern können und es wird diskutiert, wie durchführbar die genannten Angriffstechniken in der realen Welt sind. Dies hilft dabei abzuschätzen, wie relevant die Bedrohung ist, die von der RFID Technik in diesen Bereichen ausgeht. Abschließend wird noch der Aspekt der Einschränkung der Privatsphäre durch RFID Chips, besonders in Ausweisen aber auch in Produkten zur Diebstahlsicherung, besprochen und eingeordnet.
\end{abstract}

\end{document}
